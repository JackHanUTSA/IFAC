%%%%%%%%%%%%%%%%%%%%%%%%%%%%%%%%%%%%%%%%%%%%%%%%%%%%%%%%%%%%%%%%%%%%%%%%%%%%%%%%
%%%%%%%%%%%%%%%%%%%%%%%%%%%%%%%%%%%%%%%%%%%%%%%%%%%%%%%%%%%%%%%%%%%%%%%%%%%%%%%%
%%%%%%%%%%%%%%%%%%%%%%%%%%%%%%%%%%%%%%%%%%%%%%%%%%%%%%%%%%%%%%%%%%%%%%%%%%%%%%%%
%%%%%%%%%%%%%%%%%%%%%%%%%%%%%%%%%%%%%%%%%%%%%%%%%%%%%%%%%%%%%%%%%%%%%%%%%%%%%%%%
\documentclass[letterpaper, 10 pt, conference]{ieeeconf}  % Comment this line out
                                                          % if you need a4paper
%\documentclass[a4paper, 10pt, conference]{ieeeconf}      % Use this line for a4
                                                          % paper

\IEEEoverridecommandlockouts                              % This command is only
                                                          % needed if you want to
                                                          % use the \thanks command
\overrideIEEEmargins
% See the \addtolength command later in the file to balance the column lengths
% on the last page of the document
\usepackage{graphicx}          % Include this line if your 
                               % document contains figures,

\usepackage{amsmath}
\usepackage{amssymb}
\usepackage{amsfonts}
\usepackage{bbm}
\usepackage{color}
\usepackage{enumerate}	 
\usepackage{wrapfig}
\usepackage{tabularx}


\newcommand{\mathcolorbox}[2]{\colorbox{#1}{$\displaystyle #2$}}
\newtheorem{definition}{\protect\definitionname}
%\theoremstyle{plain}
\newtheorem{theorem}{\protect\theoremname}%[chapter]
\newtheorem{lemma}{\protect\lemmaname}%[chapter]
\newtheorem{assumption}{\protect\assumptionname}%[chapter]
\newtheorem{remark}{\protect\remarkname}%[chapter]
\newtheorem{corollary}{\protect\corollaryname}%[chapter]
\newtheorem{example}{\protect\examplename}%[chapter]
\providecommand{\definitionname}{Definition}
\providecommand{\theoremname}{Theorem}
\providecommand{\lemmaname}{Lemma}
\providecommand{\assumptionname}{Assumption}
\providecommand{\remarkname}{Remark}
\providecommand{\corollaryname}{Corollary}
\providecommand{\examplename}{Example}

%%%%%%%%%%%%%%%%%%%%%%%%%%%%%%%%%
\title{\LARGE \bf
Linear Integral Control for Nonlinear Systems Subject to Time-Varying Perturbations with Unknown Magnitudes
}


\author{Zhifeng Han$^{1}$, Chunjiang Qian$^{1}$, \IEEEmembership{Fellow,~IEEE}, Xiandong Chen$^{2}$ and Claire Walton$^1$
\thanks{$^{1}$Zhifeng Han and Chunjiang Qian are with Department of Electrical and Computer Engineering, The University of Texas at San Antonio, San Antonio, TX 78249, USA.
	{\tt\small zhifeng.han@utsa.edu, \tt\small chunjiang.qian@utsa.edu.}}
\thanks{$^{2}$Xiandong Chen is with School of Control Science and Engineering, Shandong University, Jinan, 250061, P.R. China and Department of Electrical and Computer Engineering, The University of Texas at San Antonio, San Antonio, TX 78249, USA.
         {\tt\small chenxiandong@hotmail.com.}}
}
%%%%%%%%%%%%%%%%%%%%%%%%%%%%%%%%%%%%%%%%%
\begin{document}

\maketitle
\thispagestyle{empty}
\pagestyle{empty}

\begin{abstract} 
This paper focuses on the problem of designing linear integral controllers for uncertain nonlinear systems subject to time-varying perturbations with unknown magnitudes by means of feedback domination method. The time-varying perturbations can include constant step disturbances, exogenous time-varying disturbances with unknown magnitudes, and modeling uncertainties with unknown system parameters. Firstly, a new linear integral controller consisting of integral dynamic is constructed to drive the states of uncertain systems without nonlinear terms to the origin asymptotically. 
Secondly, by introducing high-gain technique, a new linear integral controller is then constructed to drive the states of uncertain nonlinear systems to the origin asymptotically. Finally, based on a proposed new stability criterion, we further extend our main results to study a class of uncertain nonlinear systems with a more general time-varying perturbations with unknown magnitudes.	
\end{abstract}

\section{Introduction}\label{introduction}

This is the most critical area for "Focus Mode." In creative writing, an LLM hallucination is a funny plot twist. In robotics, a hallucination is a crash.

When controlling hardware (like your e-puck2), the "crossing paths" problem you mentioned is dangerous. If the LLM starts "thinking" about the philosophy of movement instead of outputting a motor velocity, latency spikes and the robot hits a wall.

Here is how we apply the Focus Mode architecture to robotic control to ensure safety and precision.


1. The Architecture: Split the Brain
We cannot have one LLM doing everything. We must split the control loop into two distinct agents with "Focused" system prompts.

Agent A: The Strategist (High Level)

Focus: Logic, planning, sequence, environment analysis.

Input: "Find the red object."

Output: A list of sub-goals (e.g., "Rotate 90 degrees," "Move forward 10cm").

Agent B: The Driver (Low Level - "Focus Mode")

Focus: Strictly translating sub-goals into hardware commands (JSON).

Input: "Rotate 90 degrees."

Output: {"left_motor": 500, "right_motor": -500, "duration_ms": 1000}.
%%%%%%%%%%%%%%%%%%%%%%%%%%%%%%%%%%%%%%%%%%%%%%%%%%%%%%%%%%%%%%%%%%%%%%%%%%%%%%%%%%
\section{Preliminaries}\label{preliminaries}

In order to get our main results, the following lemma is introduced firstly.
%\begin{lemma}\label{lemma 2.1}
%Let $c, ~d$ be positive constants and for any positive constant $\epsilon \geq 0$, the following inequality holds
%\begin{eqnarray}
%	|x|^c|y|^d \leq \frac{c}{c+d}\epsilon|x|^{c+d} + \frac{d}{c+d}\epsilon^{-\frac{c}{d}}|y|^{c+d}.
%\end{eqnarray}
%\end{lemma}

\begin{lemma}\label{lemma 2.2}
Consider the time-varying nonlinear system 
\begin{eqnarray}\label{s02}
\dot{x}=f(t,x),
\end{eqnarray}
where $f(t,x):\mathbb{R}_{\geq 0} \times \mathcal{D}\rightarrow \mathbb{R}^{n}$ is continuous with respect to $x$ in an open neighborhood $ \mathcal{D}\subseteq \mathbb{R}^{n}$ and piecewise continuous with respect to $t$, $f(t,0)=0,~t\in \mathbb{R}_{\geq 0}$ and the initial state is $x(0)=x_{0} \in \mathcal{D}$. If there is a continuous, positive definite function $V(x):\mathcal{D} \rightarrow \mathbb{R}_{\geq 0}$ for system (\ref{s02}) such that
	\begin{eqnarray}\label{s03}
	\dot{V}(x) \leq -\mu(t)V,
	\end{eqnarray}
where $\mu(t)$ is piecewise continuous (containing removable discontinuities and jump discontinuities)  satisfying 
	\begin{eqnarray}\label{s04}
	\lim_{t \rightarrow +\infty}\int_{0}^{t} \mu(s)ds =+\infty.
	\end{eqnarray}
Then, the trajectories of system (\ref{s02}) is locally convergent to zero.
Furthermore, if $\mathcal{D}= \mathbb{R}^{n}$, the trajectories of system (\ref{s02}) is also globally convergent to zero.
\end{lemma}
\textbf{Proof.} Firstly, integrating both sides of inequality (\ref{s03}) from $0$ to $t$, we can easily have
\begin{eqnarray*}\label{s05}
	V(t) \leq V(x_{0})e^{-\int_{0}^{t} \mu(s)ds}.
\end{eqnarray*}
From the condition (\ref{s04}), we can get that system (\ref{s02}) is locally convergent to zero. \\
%%%%%%%%%%%%%%%%%%%%%%%%%%%%
\begin{remark}\label{rem1}
Lemma \ref{lemma 2.2} extends the existing Lyapunov conditions of asymptotically stability since it will degenerate to the results when $\mu(t),~t \in \mathbb{R}_{\geq 0}$ being a negative constant, i.e., there exists a positive constant $c$ such that $\mu(t) \leq c,~t \in \mathbb{R}_{\geq 0}$. However, the criterion can allow  $\mu(t),~t \in \mathbb{R}_{\geq 0}$ is equal to zero and even positive at some time points, which makes the positive constant $c$ not exist.
We introduce the following example to further elaborate Lemma \ref{lemma 2.2}  and Remark \ref{rem1}. \\	
\end{remark}

\begin{example}\label{exam1}
Consider a scalar time-varying nonlinear system
\begin{eqnarray}\label{s06}
	\dot{x}(t)=-\mu(t)x(t),~t \in \mathbb{R}_{\geq 0},
\end{eqnarray}
where $x(0)=x_{0}>0$ and the following two cases are analyzed based on system (\ref{s06}). \\

\textbf{Case 1.} When $\mu(t)=(1+\mathrm{cos}t)$ is a continuous function with respect to $t$. Firstly, from the definition of $\mu(t)$, we get that $\mu(t)$ is equal to zero in $T_{1}=\{t| t=\pi + 2k\pi,~ t\in \mathbb{R}_{\geq 0},~ k \in \mathbb{Z}_{\geq 0}\}$ and positive in $t\in \mathbb{R}_{\geq 0}/T_{1}$. Then, we can easily obtain that  $\lim_{t \rightarrow +\infty}\int_{0}^{t} \mu(s)ds =+\infty$. Thus, we can get that the trajectories of system (\ref{s06}) is globally convergent to zero.

\textbf{Case 2.} When $\mu(t)=\left(\frac{1}{2}+\mathrm{sin}t\right)$ is a continuous function with respect to $t$.
Firstly, we get that $\mu(t)$ is a negative function in $T_{2}=\big\{t|t\in \big(\frac{7}{6}\pi + 2k\pi, \frac{11}{6}\pi + 2k\pi \big), ~ t\in \mathbb{R}_{\geq 0}, ~k \in \mathbb{Z}_{\geq 0}  \big\}$.
Then, integrating both sides of system (\ref{s06}) from $0$ to $t$, we have $\int_{0}^{t} \mu(s)ds=\left(\frac{1}{2}t-\mathrm{cos}t +1\right)$ is equal to zero at $t=0$, positive when $t>0$ and converges to $+\infty$ as $t \rightarrow +\infty$, which indicates that the trajectories of system (\ref{s06}) is globally convergent to zero.	
\end{example}




%\begin{lemma}\label{lem2}
%	For $x\in \mathbb{R}$, $y\in \mathbb{R}$, $p\geq 1$ is a constant, the following inequalities hold 
%	\begin{eqnarray*}
%		&&|x+y|^{p}\leq 2^{p-1}|x^{p}+y^{p}|,   \\[2mm]
%		&&(|x|+|y|)^{\frac{1}{p}}\leq |x|^{\frac{1}{p}}+|y|^{\frac{1}{p}}.
%	\end{eqnarray*}
%	If $p\geq 1$ is an odd integer, then 
%	\begin{eqnarray*}
%		\left|x^{\frac{1}{p}}-y^{\frac{1}{p}}\right| \leq 2^{1-\frac{1}{p}}|x-y|^{\frac{1}{p}}.
%	\end{eqnarray*}
%\end{lemma}
%
%\begin{lemma}\label{lem3}
%	Let $c,d$ be positive constant. Given any positive number $\gamma >0$, the following inequality holds 
%	\begin{eqnarray}
%		|x|^{c}|y|^{d} \leq \frac{c}{c+d}\gamma|x|^{c+d}+\frac{d}{c+d}\gamma^{-\frac{c}{d}}|y|^{c+d}.  \nonumber
%	\end{eqnarray}
%\end{lemma}

%%%%%%%%%%%%%%%%%%%%%%%%%%%%%%%%%%%%%%%%%%%%%%%%%%%%%%%%%%%%%%%%%%%%%%


\section{Main Results}\label{main}

In this paper,  a new integral controller will be constructed for system (\ref{1}). Firstly, we will modify the technique of adding an integrator to design a novel regulator for a class of chain systems, and then applying the regulator to regulate system (\ref{1}).
\subsection{Global Regulation of Chain Systems} 
\begin{theorem}\label{the1}
	For the following system
	\begin{eqnarray}\label{2}
		\left\{
		\begin{array}{l}
			\dot{z}_{i}=z_{i+1},~i=1,\ldots,n, \\
			\dot{z}_{n+1}=u,
		\end{array}
		\right.
	\end{eqnarray}
there are positive constants $k^{*}$ and $a_i,~i=1,\ldots,n$ such that for any $K(t,~z) \geq k^*$, the linear controller
	\begin{eqnarray}\label{u_z}
		u = -K(t,z)(a_1z_1  + \cdots +  a_{n}z_{n}+ z_{n+1})
	\end{eqnarray}
can globally asymptotically stabilize system (\ref{2}).
\end{theorem}
\textbf{Proof.} Based on the adding a power integrator technique, we propose a constructive method to design a controller (\ref{u_z}) and Lyapunov functions in a recursive manner. \\
\textbf{Step 1:} First, construct
\begin{eqnarray*}\label{z_v1}
	V_1 = \frac{1}{2} z_1^2.
\end{eqnarray*}
The time derivative of $V_1$ along system (\ref{2}) is
\begin{eqnarray}\label{V1_dot}
	\dot V_1|_{(\ref{2})}= z_1z_2 = \epsilon_1z_2^{*} + \epsilon_1(z_2 - z_2^{*})
\end{eqnarray}
with $\epsilon_1:=z_1$. For (\ref{V1_dot}), selecting the virtual controller $z_2^* = -\beta_1\epsilon_1=-(n+1)\epsilon_1$ yields
\begin{eqnarray}\label{V1_dot_c}
	\dot V_1|_{(\ref{2})} = -(n+1)\epsilon_1^2 + \epsilon_1(z_2 - z_2^{*}).
\end{eqnarray}
\textbf{Inductive Step:} 
Suppose at step $k-1$, there is a $C^{1}$ Lyapunov function $V_{k-1}:\mathbb{R}^{k-1}\rightarrow \mathbb{R}$, which is positive definite and proper, and a set of $C^{1}$ virtual controllers $z_{1}^{\ast },\ldots ,z_{k}^{\ast }$, defined by
\begin{eqnarray}\label{7}
\begin{array}{l}
z_{1}^{\ast }=0,
 ~~~~~~~~~~~~~~~\xi _{1}=z_{1}-z_{1}^{\ast }, \\ 
z_{2}^{\ast }=-\beta _{1}\xi _{1}, 
~~~~~~~~~\xi _{2}=z_{2}-z_{2}^{\ast }, \\ 
~~~~\vdots 
~~~~~~~~~~~~~~~~~~~~~~ \vdots \\ 
z_{k}^{\ast }=-\beta _{k-1}\xi _{k-1}, 
~~~ \xi _{k}=z_{k}-z_{k}^{\ast }
\end{array}
\end{eqnarray}
with positive constants $\beta _{1},\ldots ,\beta _{k-1}$ such that 
\begin{eqnarray}\label{8}
\begin{array}{l}
\dot{V}_{k-1}|_{(\ref{2})}
\leq 
-\left( n-k+3\right) \left( \xi _{1}^{2}+\cdots +\xi _{k-1}^{2}\right)\\
~~~~~~~~~~~~~+\xi _{k-1}\left( z_{k}-z_{k}^{\ast}\right).  
\end{array}
\end{eqnarray}
It is clear that (\ref{8}) reduces to the inequality (\ref{V1_dot_c}) when $k=2$ under the definitions of (\ref{7}). Next, we will prove (\ref{8}) also holds at step $k$. To prove this, we choose
\begin{eqnarray}\label{Wk}
	W_k=\frac{1}{2}(z_{k}-z_{k}^{*})^{2}.
\end{eqnarray}
Therefore, for $V_k = V_{k-1} + W_k$, the time derivative of the Lyapunov function $V_{k}$ along system (\ref{2})
is
\begin{eqnarray}\label{11}
\begin{array}{l}
\dot{V}_{k}|_{(\ref{2})} %\\[3mm]
%=\dot{V}_{k-1}+\dot{W}_{k}  \label{11}\\[3mm]
%\leq
%-\left( n-k+3\right) \left( \xi _{1}^{2}+\cdots +\xi_{k-1}^{2}\right)    \\[3mm]
%~~+\xi _{k-1}\left( z_{k}-z_{k}^{\ast }\right) +\frac{\partial W_{k}}{\partial z_{k}}\dot{z}_{k}+\sum_{l=1}^{k-1}\frac{\partial W_{k}}{\partial z_{l}}\dot{z}_{l} \\[3mm]
%\leq
%-\left( n-k+3\right) \left( \xi _{1}^{2}+\cdots +\xi_{k-1}^{2}\right)   \\[3mm]
%~~+\xi _{k-1}\left( z_{k}-z_{k}^{\ast }\right)+\xi _{k}z_{k+1}+\sum_{l=1}^{k-1}\frac{\partial W_{k}}{\partial z_{l}}\dot{
%	z}_{l}  \\[3mm]
\leq
-\left( n-k+3\right) \left( \xi _{1}^{2}+\cdots +\xi_{k-1}^{2}\right)   \\[3mm]
~~~~~~~~~~+\xi _{k-1}\left( z_{k}-z_{k}^{\ast }\right)+\xi _{k}z_{k+1}^{\ast }+\sum_{l=1}^{k-1}\frac{\partial W_{k}}{\partial z_{l}}\dot{z}_{l}\\[3mm]
~~~~~~~~~~+\xi _{k}\left( z_{k+1}-z_{k+1}^{\ast }\right).
\end{array}
\end{eqnarray}

Now we give the estimate of the terms $\xi _{k-1}\left( z_{k}-z_{k}^{\ast }\right) $ and $\sum_{l=1}^{k-1}\frac{\partial W_{k}}{\partial z_{l}}\dot{z%
}_{l}$ in (\ref{11}).

Firstly, from (\ref{7}) and by means of the Young inequality, we have 
\begin{eqnarray} \label{12}
	\xi _{k-1}\left( z_{k}-z_{k}^{\ast }\right) =\xi _{k-1}\xi _{k}\leq \frac{1}{%
		2}\xi _{k-1}^{2}+\frac{1}{2}\xi _{k}^{2}. 
\end{eqnarray}
Next, from (\ref{7}), and
%\begin{eqnarray}\label{13}
%W_{k}& =&\frac{1}{2}\left( z_{k}-z_{k}^{\ast }\right) ^{2}.
%&=&\frac{1}{2}\left( z_{k}+\beta _{k-1}\xi _{k-1}\right) ^{2}  \notag \\
%&\vdots&  \notag \\
%&=&\frac{1}{2}\left( z_{k}+\beta _{k-1}\left( z_{k-1}+\cdots +\beta
%_{2}\left( z_{2}-z_{2}^{\ast }\right) \cdots \right) \right) ^{2}  \notag \\
%&=&\frac{1}{2}\left( z_{k}+\beta _{k-1}\left( z_{k-1}+\cdots +\beta
%_{2}\left( z_{2}+\beta _{1}z_{1}\right) \cdots \right) \right) ^{2}.  \notag
%\end{eqnarray}
 by means of (\ref{Wk}), we have
\begin{eqnarray}\label{14}
\sum_{l=1}^{k-1}\frac{\partial W_{k}}{\partial z_{l}}\dot{z}_{l}
%	=\sum_{l=1}^{k-1}\frac{\partial W_{k}}{\partial z_{l}}z_{l+1}   \\
%&=&\left( z_{k}-z_{k}^{\ast }\right) \sum_{l=1}^{k-1}\beta _{k-1}\cdots
%	\beta _{l}z_{l+1}  \notag \\
%&=&\left( z_{k}-z_{k}^{\ast }\right) \sum_{l=1}^{k-1}\beta _{k-1}\cdots
%	\beta _{l}\left( \left( z_{l+1}-z_{l+1}^{\ast }\right) +z_{l+1}^{\ast
%	}\right)   \notag \\
%&=&\xi _{k}\sum_{l=1}^{k-1}\beta _{k-1}\cdots \beta _{l}\left( \xi_{l+1}-\beta _{l}\xi _{l}\right)   \notag \\
\leq 
\frac{1}{2}\left( \xi _{1}^{2}+\cdots +\xi _{k-1}^{2}\right)+c_{k}\xi _{k}^{2},
\end{eqnarray}%  
where $c_{k}$ is a positive constant. Substituting (\ref{12}) and (\ref{14}) into (\ref{11}) yields
\begin{eqnarray}\label{15} 
\begin{array}{l}
\dot{V}_{k}|_{(\ref{2})}
%         &\leq &-\left( n-k+3\right) \left( \xi _{1}^{2}+\cdots +\xi
%        _{k-1}^{2}\right) +\left( z_{k}-z_{k}^{\ast }\right) z_{k+1}^{\ast } \\
\leq 
 -\left( n-k+2\right) \left( \xi _{1}^{2}+\cdots +\xi_{k-1}^{2}\right)   \\[3mm]
~~~~~~~~~~+\xi _{k}z_{k+1}^{\ast }+\left( c_{k}+\frac{1}{2}\right) \xi _{k}^{2}\\[3mm]
~~~~~~~~~~+\left( z_{k}-z_{k}^{\ast}\right) \left( z_{k+1}-z_{k+1}^{\ast }\right). 
\end{array}
\end{eqnarray}
Now construct the virtual controller%
\begin{eqnarray}\label{16}
\begin{array}{l}
z_{k+1}^{\ast } =-\beta _{k}\xi _{k} %\\[3mm]
=-\left( c_{k}+\frac{1}{2}+n-k+2\right) \xi _{k}
\end{array}
\end{eqnarray}
and substituting (\ref{16})  into (\ref{15}), we have
\begin{eqnarray*}\label{17}
\begin{array}{l}
\dot{V}_{k}|_{(\ref{2})}
\leq
-\left( n-k+2\right) \left( \xi _{1}^{2}+\cdots +\xi_{k-1}^{2}+\xi _{k}^{2}\right) \\[3mm]
~~~~~~~~~~+\left( z_{k}-z_{k}^{\ast }\right) \left(z_{k+1}-z_{k+1}^{\ast }\right).  
\end{array}
\end{eqnarray*}
This completes the inductive proof. \\
\\
\textbf{Last Step:} 
The inductive argument shows that (\ref{8}) holds for $k=n+1$ with a set of virtual controllers (\ref{7}). Based on the inductive argument, we can choose the $(n+1)$th Lyapunov function    
\begin{eqnarray*}\label{18}
\begin{array}{l}
V_{n+1} =V_{n}+W_{n+1}   %\\[3mm]
=V_{n}+\frac{1}{2}\left( z_{n+1}-z_{n+1}^{\ast }\right) ^{2}.
%&=&V_{n}+\frac{1}{2}\xi _{n+1}^{2}.  \notag
\end{array}
\end{eqnarray*}
The time derivative of $V_{n+1}$ along system (\ref{2}) is
\begin{eqnarray}\label{19}
\begin{array}{l}
\dot{V}_{n+1}|_{(\ref{2})} 
%&=&\dot{V}_{n}+\dot{W}_{n+1}   \\
%&\leq &-2\left( \xi _{1}^{2}+\cdots +\xi _{n}^{2}\right) +\xi _{n}\left(
%z_{n+1}-z_{n+1}^{\ast }\right)  \notag \\
%&&+\left( z_{n+1}-z_{n+1}^{\ast }\right) \left( z_{n+1}-z_{n+1}^{\ast}\right) ^{^{\prime }}  \notag \\
\leq 
-2\left( \xi _{1}^{2}+\cdots +\xi _{n}^{2}\right) +\xi_{n+1} u  \\[3mm]
~~~~~~~~~~~~~+\xi _{n}\xi _{n+1}+\left( z_{n+1}-z_{n+1}^{\ast }\right) \left( -\dot{z}_{n+1}^{\ast }\right). 
\end{array}
\end{eqnarray} 
Similar to the estimate of the terms (\ref{12}) and (\ref{14}), we have
\begin{eqnarray}\label{20}
	\xi _{n}\xi _{n+1}\leq \frac{1}{2}\xi _{n}^{2}+\frac{1}{2}\xi _{n+1}^{2}
\end{eqnarray}
and
\begin{eqnarray} \label{21}
\begin{array}{l}
\left( z_{n+1}-z_{n+1}^{\ast }\right) \left( -\dot{z}_{n+1}^{\ast }\right)
\leq 
\frac{1}{2}\left( \xi _{1}^{2}+\cdots +\xi _{n}^{2}\right) \\[3mm]
~~~~~~~~~~~~~~~~~~~~~~~~~~~~~~~~+c_{n+1}\xi_{n+1}^{2}. 
\end{array}
\end{eqnarray}
Substituting (\ref{20}) and (\ref{21}) into (\ref{19}), we have
\begin{eqnarray}\label{22}
\begin{array}{l}
\dot{V}_{n+1}|_{(\ref{2})} 
\leq
-\left( \xi _{1}^{2}+\cdots +\xi _{n}^{2}\right) +\left( c_{n+1}+\frac{1}{2}\right) \xi_{n+1}^{2}\\[3mm]
~~~~~~~~~~~~~~+\xi_{n+1} u. 
\end{array}
\end{eqnarray}

By the adding a power integrator technique, we can simply choose the following controller
\begin{eqnarray}\label{23}
u =-K(t,z)\xi _{n+1},
\end{eqnarray}
where $K(t,z)\geq k^{*}=c_{n+1}+\frac{3}{2}$.
Substituting (\ref{23}) into (\ref{22}), we have
\begin{eqnarray}\label{24}
	\dot{V}_{n+1}|_{(\ref{2})}\leq -\left( \xi _{1}^{2}+\cdots +\xi _{n}^{2}+\xi
	_{n+1}^{2}\right).  
\end{eqnarray}
Thus, we have achieved that system (\ref{2}) is globally asymptotically
stable under controller (\ref{23}), that is 
\begin{eqnarray*} \label{25}
\begin{array}{l}
u =-K(t,z)\xi _{n+1}\\[3mm]
%&=&-K(t,z)\left( z_{n+1}-z_{n+1}^{\ast }\right)  \notag \\
%&=&-K(t,z)\left( z_{n+1}-\left( z_{n+1}^{\ast }\right) \right)  \notag\\
%% &=&-K(t,z)\left( z_{n+1}-\left( -\beta _{n}\xi _{n}\right) \right) 
%% \notag \\
%% &=&-K(t,z)\left( z_{n+1}+\beta _{n}\xi _{n}\right)  \notag \\
%&\vdots&  \notag \\
~~=-K(t,z)( z_{n+1}+\beta_{n}( z_{n}+\cdots +\beta_{2}( z_{2}%\\[3mm]
+\beta_{1}z_{1}) \cdots)),
\end{array}
\end{eqnarray*}    
where $a_{i}=\beta _{i}\cdots\beta _{n},~i=1,\ldots,n$. This completes our proof.\\
\\
Now it is time to present our main results for global regulating system (\ref{1}). First, we utilize Theorem \ref{the1} to solve the regulation problem of chain system (\ref{1}) when $f_i (\cdot)= 0, ~i = 1,\ldots,n$, i.e,
\begin{eqnarray}\label{exp1}
	\left\{
	\begin{array}{l}
		\dot{x}_{i}=x_{i+1},~i=1,\ldots,n-1,   \\
		\dot{x}_{n}=u+d\left( t,x\right). 
	\end{array}
	\right.
\end{eqnarray}
To solve the problem, we assume that the time-varying perturbations with unknown magnitudes $d(t, x)$ satisfies the following assumption.
% assumption 3.1
\begin{assumption}\label{assum1}
Assume there are an unknown constant $\theta $ and a known function $\alpha(t, x) \geq 1$ such that
\begin{eqnarray*}
	d(t, x) = \theta\alpha(t,x).
\end{eqnarray*}
\end{assumption}
\begin{remark}
The uncertain function $d(t,x)$ satisfying Assumption \ref{assum1} encompasses several types of uncertainties in system (\ref{exp1}). First, it includes constant disturbances as its special case when $\alpha(t,x) = 1$. For exogenous time-varying disturbance such as $d(t,x) = c(1 + 0.5\mathrm{sin}(t))$ with unknown magnitude $c$, we can simply choose $\alpha(t,x) = 2(1+ 0.5\mathrm{sin}(t)) \geq 1$ and $\theta = c/2 $. Moreover, $d(t,x)$ can include internal modeling uncertainties such as $d(t,x)=\theta(1+x_2^2)$ with unknown system parameter $\theta$. 	
\end{remark}
%\textbf{Theorem 3.1.2}
\begin{theorem}\label{the2}
Under Assumption \ref{assum1}, there are positive constants $k^{*}$ and $a_i,~i=1,\ldots,n$, such that the following  integral controller
\begin{eqnarray}\label{t1}
	\left\{
	\begin{array}{l}
		u = -k^*\alpha(t,x)(a_1x_0 +a_2x_1+ \cdots \\
		~~~~~+ a_{n}x_{n-1} + x_n ), \\
		\dot x_0 = x_1
	\end{array}
	\right.
\end{eqnarray}
globally regulates system (\ref{exp1}).
\end{theorem}
\textbf{Proof.} Define $z_1 = x_0 - \frac{\theta}{k*a_1}$, $z_i=x_{i-1}, i = 2, \ldots, n+1$. Under the new coordinates, it is clear that the closed-loop system (\ref{exp1}) and (\ref{t1}) can be rewritten as
\begin{eqnarray}\label{exp1_1} 
\dot{z}&=&\left[ 
\begin{array}{c}
		z_{2} \\ 
		\vdots \\ 
		z_{n+1} \\ 
		-k^*\alpha \left( t,x\right) \left( a_{1}z_{1}+\cdots
		+a_{n}z_{n}+z_{n+1}\right)
	\end{array}
	\right]  \nonumber\\
	&=&F\left( t,z\right).        
\end{eqnarray}
By the proof of Theorem \ref{the1}, we can find positive constants $k^*$ and $a_i,~i=1,\ldots,n$ such that for $K(t,z) = k^* \alpha(t,x) \geq k^*$, system (\ref{exp1_1}) is globally regulated. In the case when $d(t,x) = \theta$ is an unknown constant, Theorem \ref{the2} holds for a controller with constant gains.
\begin{corollary}\label{cor1}
The system (\ref{exp1}) with $d(t,x) = \theta$ being an unknown constant $\theta$ can be globally regulated by the integral controller
\begin{eqnarray*}\label{t2}
	\left\{
	\begin{array}{l}
		u = -k^*(a_1x_0 +a_2x_1+ \cdots + a_{n}x_{n-1} + x_n ), \\
		\dot x_0 = x_1
	\end{array}
	\right.
\end{eqnarray*}
for appropriate positive constants $k^*$ and $a_i,~i=1,\ldots,n$.
\end{corollary}
\subsection{Global Regulation of Nonlinear Systems}
This section considers the global regulation problem of system (\ref{1}) when system nonlinearities satisfy the following assumption.
%Assumptions 3.2
\begin{assumption}\label{assum2} 
For $i=1,\ldots ,n$, there is a known constant $c$ such that
\begin{eqnarray*}\label{31}
	\left\vert f_{i}\left( x_{1},\ldots ,x_{i}\right) \right\vert \leq c\left(\left\vert x_{1}\right\vert +\cdots +\left\vert x_{i}\right\vert \right).  
\end{eqnarray*}
\end{assumption}
\begin{theorem}\label{the3}
Under Assumptions \ref{assum1} and \ref{assum2}, there are positive constants $k^*$ and $a_1, \ldots, a_n$ such that for a large enough constant $L\geq 1$, the following integral controller
\begin{eqnarray} \label{32}
\left\{
\begin{array}{l}
u =-L^{n}k^*\alpha \left( t,x\right) \big(a_{1}x_{0}+a_{2}x_{1}+a_{3}\frac{x_{2}}{L} \\[3mm]
~~~~~+\cdots+a_{n}\frac{x_{n-1}}{L^{n-2}}+\frac{x_{n}}{L^{n-1}}\big),   \\[3mm]
\dot{x}_{0} = Lx_{1}  
\end{array}
\right.
\end{eqnarray}
with $\alpha(t,x)$ defined in Assumption \ref{assum1} solves the global regulation problem of system (\ref{1}).
\end{theorem}
\textbf{Proof.} Define $z_1 = x_0 - \frac{\theta}{k*a_1L^n}$, $z_{i}=\frac{x_{i-1}}{L^{i-2}},~ i = 2, \ldots, n+1$. By choosing the same constants $k^*$ and $a_i,~i =1, \ldots, n$ as in Theorem \ref{the2}, the closed-loop system (\ref{1}) and (\ref{32}) under the new coordinates can be rewritten as
\begin{eqnarray*}\label{33}
\left\{
\begin{array}{l}
\dot{z}_{1}=Lz_{2},\\ 
\dot{z}_{2}=Lz_{3}+f_{1}(x_{1}),\\ 
\dot{z}_{3}=Lz_{4}+\frac{f_{2}(x_{1},x_{2})}{L},\\ 
~~~~~\vdots  \\ 
\dot{z}_{n}=Lz_{n+1}+\frac{f_{n-1}(x_{1},\ldots,x_{n-1})}{L^{n-2}},\\ 
\dot{z}_{n+1}=-Lk^*\alpha \left( t,x\right) \left( a_{1}z_{1}+\cdots
+a_{n}z_{n}+z_{n+1}\right)\\
~~~~~~~~~+\frac{f_{n}(x_{1},\ldots,x_{n})}{L^{n-1}},\\ 
\end{array}
\right.
\end{eqnarray*}
which can be further rewritten as the following matrix form
\begin{eqnarray}\label{34}
	\dot{z}=LF\left( t,z\right) +\varPhi,
\end{eqnarray}
where $F(t,z)$ is the same as the one in (\ref{exp1_1}) and $\varPhi=\big(0,f_{1}\left( x_{1}\right),\frac{f_{2}\left( x_{1},x_{2}\right) }{L},\ldots,\frac{f_{n}\left( x_{1},\ldots ,x_{n}\right) }{L^{n-1}}\big)^{T}$. 




By using the same Lyapunov function  $V_{n+1}$ constructed in Theorem \ref{the1}, we can see that the time derivative of $V_{n+1}$ along system (\ref{34}) is
\begin{eqnarray}\label{35}
\begin{array}{l}
\dot{V}_{n+1}|_{(\ref{34})} \\[3mm]
=\frac{\partial V_{n+1}}{\partial z}\left( LF\left( t,z\right) +\varPhi \right)\\[3mm]
%=L\frac{\partial V_{n+1}}{\partial z}F\left( t,z\right) +\sum_{i=2}^{n+1}
%	\frac{\partial V_{n+1}}{\partial z_{i}}\frac{f_{i-1}\left( x_{1},\ldots
%		,x_{i-1}\right) }{L^{i-2}}  \\[3mm]
\leq 
-L\sum_{l=1}^{n+1}\xi _{l}^{2}+\sum_{i=2}^{n+1}\sum_{l=i}^{n+1}\frac{\partial V_{l}}{\partial z_{i}}\frac{f_{i-1}\left( x_{1},\ldots,x_{i-1}\right)}{L^{i-2}}. 
\end{array}
\end{eqnarray}
By Assumption \ref{assum2} and the fact $L\geq 1$, we have  
\begin{eqnarray}\label{36}
\begin{array}{l}
\left\vert \frac{f_{i-1}(\cdot) }{L^{i-2}}\right\vert  
\leq 
\frac{c\left( \left\vert x_{1}\right\vert+\cdots +\left\vert x_{i-1}\right\vert \right) }{L^{i-2}}   %\\[3mm]
%~~~~~~~~~~
%\leq 
%c\left( \left\vert z_{2}\right\vert +\cdots +\left\vert z_{i}\right\vert \right) \\[3mm]
%~~~~~~~~~~
\leq 
\tilde{c}\left( \left\vert \xi _{1}\right\vert +\cdots +\left\vert \xi_{i}\right\vert \right) 
\end{array}
\end{eqnarray}
for a positive constant $\tilde{c}$.
In addition, from the definition of system (\ref{13}), we have
\begin{eqnarray} \label{37}
\begin{array}{l}
\sum_{l=i}^{n+1}\frac{\partial V_{l}}{\partial z_{i}}\leq \bar{c}%
\sum_{l=i}^{n+1}\left\vert \xi _{i}\right\vert
\end{array}  
\end{eqnarray}
for a positive constant $\bar{c}$.
Then, by substituting (\ref{36}) and (\ref{37}) into (\ref{35}), we have%
\begin{eqnarray*} \label{38}
\begin{array}{l}
\dot{V}_{n+1}|_{(\ref{34})} % \\[3mm]
%\leq 
%-L\sum_{l=1}^{n+1}\xi_{l}^{2} \\[3mm]
%~~+\sum_{i=2}^{n+1}\bar{c}\sum_{l=i}^{n+1}\left\vert \xi_{i}\right\vert \tilde{c}\left( \left\vert \xi _{1}\right\vert +\cdots+\left\vert \xi _{i}\right\vert \right)  \\[3mm]
\leq 
-L\sum_{l=1}^{n+1}\xi _{l}^{2}+\hat{c}\sum_{l=1}^{n+1}\xi _{l}^{2}
\end{array}
\end{eqnarray*}%
for a positive constant $\hat{c}$.

Selecting a large enough $L \geq \hat{c}+1$, we can get a relation same as (\ref{24}). As a result, the integral controller (\ref{32}) can regulate system (\ref{1}) under Assumptions \ref{assum1} and \ref{assum2}.
%\textbf{Remark 3.2.1}
%\begin{remark}\label{chrm1}
%In this section, a static high-gain parameter $L$ is introduced to achieve the global regulation problem for system (\ref{1}) under Assumptions \ref{assum1} and \ref{assum2}. The aim of introducing $L$ is to tackle the nonlinear terms $ f_i(\cdot),~i=1,\ldots,n$ by expanding the gain parameters $k^{*}$ and $a_i,~i=1,\ldots,n$ given in Theorem \ref{the2}, $L$ times, which makes the proof of Theorem \ref{the3} more concise. On the other hand, we can also achieve the global regulation of system (\ref{1}) under Assumptions \ref{assum1} and \ref{assum2} by following the proof of Theorems \ref{the1} and \ref{the2}. Specifically speaking, we can estimate the nonlinear term in each step of the proof process of Theorem  \ref{the1}, and then re-select gain parameters $k^{*}$ and $a_i,~i=1,\ldots,n$, rather than introducing $L$, to achieve our control objective.
%\end{remark}
%In order to elaborate Remark \ref{chrm1}, the following example is introduced.
%\textbf{Example 3.1}
%\textbf{Example 3.2.1}





\subsection{Extension}
We reconsider system (\ref{1}) where system vanishing uncertainties $f_{i}(x_{1},\ldots,x_{i}),~i=1,2,\ldots,n$ satisfy Assumption  \ref{assum2} and the non-vanishing uncertainty $d(t, x) $ satisfies the following assumption.

%\textbf{Assumption 3} 
\begin{assumption}\label{assum3}
%
Assume the non-vanishing uncertainty $d(t, x)$ satisfies
\begin{eqnarray*}
	d(t, x) = \theta\alpha(t,x),
\end{eqnarray*}
where $\theta $ is an unknown constant and $\alpha(t, x)$ is a continuous function. Moreover,  $\alpha(t, x)$ also satisfies
$\alpha(t, x) \geq \bar{\alpha}(t) \geq 0$ where the continuous function $\bar{\alpha}(t)$ is a periodic function with $T>0$ being the period.
%where $\theta $ is an \textit{unknown} constant and $\alpha(t, x)$ is a continuous function. Moreover,  $\alpha(t, x)$ also satisfies
%$\alpha(t, x) \geq \bar{\alpha}(t)$ where the continuous function $\bar{\alpha}(t)$ is a periodic function and satisfies $\int_{0}^{T}\bar{\alpha}(s)> \sigma $ with $T>0$ being the period of $\bar{\alpha}(t)$ and $\sigma$ being a positive constant.
\end{assumption}




\begin{remark}
Assumption \ref{assum3} is significantly different from Assumption \ref{assum1}, and brings great difficulties to the stability analysis of system (\ref{1}). Specifically speaking, due to the condition $\alpha(t, x) \geq 1$ in Assumption \ref{assum1}, there must be a constant $k^*$ in Theorem \ref{the2} such that $k^*\alpha(t, x) > c_{n+1}+\frac{1}{2}$ for all $(t, x) \in \mathbb{R}_{\geq 0} \times \mathbb{R}^{n}$. However, since $\alpha(t, x) \geq \bar{\alpha}(t) \geq 0$ holds in Assumption \ref{assum3} and for all $(t, x) \in \mathbb{R}_{\geq 0} \times \mathbb{R}^{n}$, we cannot choose the constant $k^*$, such that $k^*\alpha(t, x) > c_{n+1}+\frac{1}{2}$, which brings great challenges to the stability analysis of the closed-loop system.
\end{remark}

Based on Assumptions \ref{assum2} and \ref{assum3}, we can have the following result.
%\textbf{Theorem 5.1} 
\begin{theorem}\label{the4}
%
Under Assumptions \ref{assum2} and \ref{assum3}, there are positive constants $k^*$ and $a_1, \ldots, a_n$ such that for a large enough constant $L\geq 1$, the following integral controller
\begin{eqnarray} \label{e32}
\left\{
\begin{array}{l}
u =-L^{n}k^* \alpha(t,x) \big(a_{1}x_{0}+a_{2}x_{1}+\cdots+a_{n}\frac{x_{n-1}}{L^{n-2}}\\[3mm]
~~~~~+\frac{x_{n}}{L^{n-1}}\big),  \\[3mm]
\dot{x}_{0} = Lx_{1}  
	\end{array}
	\right.
\end{eqnarray}
with $\alpha(t,x)$ defined in Assumption \ref{assum3} solves the global regulation problem of system (\ref{1}).
\end{theorem}
\textbf{Proof.} Following the proof of Theorems \ref{the1}, \ref{the2} and \ref{the3}, we can easily obtain 
\begin{eqnarray}\label{e1} 
	\dot{V}_{n+1}|_{(\ref{34})} 
	\leq -L\sum_{l=1}^{n}\xi _{l}^{2}+C\xi _{n+l}^{2} +\xi _{n+1}u,
\end{eqnarray}
where $C$ is a positive constant independent of $L$. From (\ref{e32}), we construct the controller 
\begin{eqnarray}\label{e2} 
	u=-L^{2}k^{*}\alpha(t, x)\xi _{n+1}
\end{eqnarray}
and substituting (\ref{e2}) into (\ref{e1}), we have
\begin{eqnarray}\label{e3} 
	\dot{V}_{n+1}|_{(\ref{34})} 
	\leq -L\sum_{l=1}^{n}\xi _{l}^{2}+C\xi _{n+1}^{2} -L^{2}k^{*}\alpha(t, x)\xi _{n+1}^{2},
\end{eqnarray}
and by means of Assumption \ref{assum3} and $L\geq 1$, we further obtain
\begin{eqnarray}\label{e4} 
	\dot{V}_{n+1}|_{(\ref{34})} 
	\leq 
	-L\sum_{l=1}^{n}\xi _{l}^{2}-(Lk^{*}\bar{\alpha}(t)-C)\xi _{n+l}^{2}.
\end{eqnarray}
Then, from Assumption \ref{assum3}, we have $\bar{\alpha}(t) \geq 0$. If $\bar{\alpha}(t) > 0$ for $t\in[0,T]$, and following the continuous of  $\bar{\alpha}(t)$ and the proof of Theorem \ref{the3}, we can easily prove Theorem \ref{the4}.

If there exists a time $t^{*}\in [0,T]$ such that $\bar{\alpha}(t^{*}) = 0$ and without loss of generality, we assume that only $t^{*}\in [0,T]$ exists making $\bar{\alpha}(t^{*}) = 0$. Firstly, define $T_{1}=\{\bar{\alpha}(t) \leq \varepsilon | ~t\in [0,T]\}$, $T_{2}=\{\varepsilon< \bar{\alpha}(t) \leq L | ~t\in [0,T]\}$ and $T_{3}=\{\bar{\alpha}(t) \geq L | ~t\in [0,T]\}$, where $\varepsilon$ is a small positive constant and $T_{3}$ can be an empty set. We can get that $t^{*} \in T_{1}$ and form (\ref{e4}), we have
\begin{eqnarray}\label{e5} 
	\dot{V}_{n+1}|_{(\ref{34})} 
	\leq 
	\left\{
	\begin{array}{l}
		C\sum_{l=1}^{n+1}\xi _{l}^{2},~~~~~~~~~~~~~~~~~~~t\in T_{1},\\
		-(Lk^{*}\bar{\alpha}(t)-C)\sum_{l=1}^{n+1}\xi _{l}^{2}, ~t\in T_{2},\\
		-L\sum_{l=1}^{n+1}\xi _{l}^{2}, ~~~~~~~~~~~~~~~~~t\in T_{3}.\\
	\end{array}
	\right.
\end{eqnarray}
By means of $\varepsilon< \bar{\alpha}(t) \leq L$ for $t \in T_{2}$ and from (\ref{7}), (\ref{e5}) can be further rewritten as
\begin{eqnarray}\label{e6} 
	\dot{V}_{n+1}|_{(\ref{34})} 
	\leq 
	\left\{
	\begin{array}{l}
		2CV_{n+1},~~~~~~~~~~~~~~~~~t\in T_{1},\\
		-2(Lk^{*}\varepsilon-C)V_{n+1}, ~~~t\in T_{2},\\
		-2LV_{n+1}, ~~~~~~~~~~~~~~~t\in T_{3}.\\
	\end{array}
	\right.
\end{eqnarray}

Integrating both sides of inequality (\ref{e6}) from $0$ to $T$, we have
\begin{eqnarray}\label{e7}
	V(T) 
	\leq 
	V(0) e^{2\left(\int_{T_{1}}Cds-\int_{T_{2}}(Lk^{*}\varepsilon-C)ds-\int_{T_{3}}Lds\right)}.
\end{eqnarray}
From (\ref{e7}), we  can get a large $L\geq 1$, such that $Lk^{*}\varepsilon-C>0$ and $\int_{T_{1}}Cds<\int_{T_{2}}(Lk^{*}\varepsilon-C)ds$ hold, which indicates $V(T) \leq V(0) $. On the other hand, since $\bar{\alpha}(t)$ is a periodic function with $T>0$ being the period, we can eventually get $\lim_{t \rightarrow +\infty}V(t)=0$. By Lemma \ref{lemma 2.2}, we can achieve the proof of Theorem \ref{the4}.










%\begin{eqnarray}\label{s6}
%\dot{x}=
%\left[
%\begin{array}{l}
%	x_{2}^{p}\\
%	x_{3}\\
%	-a_{3}\left(a_{1}x_{1}+a_{2}x_{2}^{\frac{p+1}{2}}+x_{3}\right)^{\frac{2p}{p+1}}
%\end{array}
%\right]
%=H(t,x).
%\end{eqnarray}


\section{Examples}\label{EX}

\begin{example}\label{exam2}
To show the feasibility of the proposed strategy, we first consider the $2$-dimension case when the disturbance is a constant in system (\ref{exp1}), i.e., $d(t,x) = \theta = 2.$ By Corollary \ref{cor1}, an integral controller is constructed as
\begin{eqnarray}\label{4.1}
	\left\{
	\begin{array}{l}
		u = -k^*(a_1x_0 + a_2x_1+ x_2  ) ,\\
		\dot x_0 = x_1,
	\end{array}
	\right.
\end{eqnarray}	
where $k^* = 3, ~a_1=1, ~a_2 = 2$ and the initial condition is $(x_0(0), x_1(0),x_2(0)) = (1, -1, 1.5)$. From the simulation result shown in Figure \ref{fig1}, we can see that the states $x_1$ and $x_2$ will converge to zero asymptotically and $x_0$ will converge to the constant $\frac{\theta }{a_1k^*}=\frac{2}{3}$.

\begin{figure}
	\centering
	\includegraphics[height=2in,width=3.5in]{./images/fig1.pdf}
	\caption{Trajectories of (\ref{exp1}) and (\ref{4.1}) with $d(t,x)=2$}
	\label{fig1}
\end{figure}
When $d(t,x)$ is a time-varying function with an unknown magnitude, for example $d(t,x) = \theta(1 + 0.5\mathrm{sin}(t))$, the controller (\ref{4.1}) with constant gain will not be sufficient to drive the output to zero. As a matter of fact, as shown in the simulation in Figure \ref{fig2} under $\theta = -2$ and the  same initial condition, there are oscillations even for a large $k^*$.

\begin{figure}
	\centering
	\includegraphics[height=2in,width=3.5in]{./images/fig2.pdf} 
	\caption{Trajectories of (\ref{exp1}) and (\ref{4.1}) with time-varying $d(t,x)$}
	\label{fig2}
\end{figure}

By Theorem \ref{the2}, we construct an integral controller 
\begin{eqnarray}\label{4.2}
	\left\{
	\begin{array}{l}
		u = -k(t,x)(a_1x_0  + a_2x_1 + x_2 ), \\
		\dot x_0 = x_1
	\end{array}
	\right.
\end{eqnarray}
with a time-varying gain $k(t,x) = 6(1+0.5\mathrm{sin}(t)) \geq k^* = 3$.

Based on the same initial condition and $\theta = -2$ , the state trajectories of closed-loop system (\ref{exp1}) and (\ref{4.2}) are shown in Figure \ref{fig3}. Clearly the states $x_1$ and $x_2$ of the closed-loop system converge to zero asymptotically.
\begin{figure}
	\centering
	\includegraphics[height=2in,width=3.5in]{./images/fig3.pdf}
	\caption{Trajectories of (\ref{exp1}) and (\ref{4.2}) with time varying $d(t,x)$}
	\label{fig3}
\end{figure}
\end{example}



Next, we consider a system with both a vanishing uncertainty and a non-vanishing uncertainty.
\begin{example}\label{exam3}
	Consider the following system inspired by the pendulum dynamic
	\begin{eqnarray}\label{4.3}
		\left\{
		\begin{array}{l}
			\dot x_1 = x_2, \\
			\dot x_2 = u + \theta(1+x_2^{2}) + \mathrm{sin}(x_1)\delta(t),
		\end{array}
		\right.
	\end{eqnarray}
where $\theta$ is an unknown constant and $\delta(t)$ is an unknown disturbance satisfying $|\delta(t)| \leq 1$.
\end{example}
Firstly, we can verify that
\begin{eqnarray*}\label{4.4}
	|\mathrm{sin}(x_1)\delta(t)|\leq |x_1|,
\end{eqnarray*}
which satisfies Assumption \ref{assum2}. By Theorem \ref{the3}, we can construct the following integral controller
\begin{eqnarray}\label{4.5}
	\left\{
	\begin{array}{l}
		u = -L^2k^*(1+x_2^{2})(a_1x_0 + \frac{x_2}{L} + a_2x_1  ), \\
		\dot x_0 = Lx_1.
	\end{array}
	\right.
\end{eqnarray}
In the simulation shown in Figure \ref{fig4}, we chose $\delta(t) = \mathrm{sin}(t),~ \theta = 3, ~L = 2,~ k^* = 3,~ a_1 = 1,~ a_2 = 2$ and the initial conditions $(x_0(0), x_1(0),x_2(0)) = (1, -1, 1.5)$. Clearly the states $x_1$ and $x_2$ of the closed-loop system converge to zero asymptotically.
\begin{figure}
	\centering
	\includegraphics[height=2in,width=3.5in]{./images/fig4.pdf}
	\caption{Trajectories of (\ref{4.3}) and (\ref{4.5}) with time varying $d(t,x)$}
	\label{fig4}
\end{figure}

%%\textbf{Remark 4.1} 
%\begin{remark}\label{rem4}
%	As demonstrated in Figures \ref{fig1}, \ref{fig3} and \ref{fig4}, the states $x_1$, $x_2$ convergence to zero asymptotically regardless of various formats of the uncertainties. The unknown constant $\theta$ in the step disturbance, time-varying disturbance, or system uncertainty can also be recovered from the final value of the integral state $x_0$ guaranteed by Theorem \ref{the2} or Theorem \ref{the3}.
%\end{remark}





\section{Conclusion}
This paper has presented a new method to design a integral controller to regulate the states of a class of uncertain nonlinear systems. Compared to the traditional PID controller, our proposed controller can handle more general uncertainties beyond constant step disturbance, such as external time-varying disturbances with unknown magnitudes and internal modeling uncertainties due to unknown parameters. Moreover, the system states will converge to the origin and the unknown magnitude/parameter can be recovered from the integral state. 
%Specifically speaking, S \ref{preliminaries} proposes a new stability criterion for the stability analysis of a class of nonlinear systems subject to time-varying perturbations with unknown magnitudes. Chapter \ref{main} constructs integral controllers for a class of chain and nonlinear systems to achieve global regulation problems further more we analyzes a more general condition on  the time-varying perturbations with unknown magnitudes. Chapter \ref{EX} proposes several example to show the feasibility of the proposed strategies.

%%%%%%%%%%%%%%%%%%%%%%%%%%%%%%%%%%%%%%%%%%%%%%%%%%%%%%%%%%%%%%%%%%%%%%%%%%%%%%%%

\bibliographystyle{IEEEtran}        
\bibliography{references}  
\end{document}
