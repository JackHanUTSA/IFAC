
\section{Conclusion}
This work introduces the \textbf{Dynamic Penetrative Trajectory Adaptation (DPTA)} framework, a novel architecture bridging 
high-level semantic planning and low-level dynamic control in embodied agents. While LLMs provide broad world knowledge 
capable of decomposing abstract tasks, they lack real-world experience and physical grounding, often generating plans that 
are plausible but not executable. DPTA addresses these limitations by synthesizing \textit{Penetrative AI}~\cite{xu2024penetrative}, \textit{Dynamic Movement Primitives (DMP)}~\cite{dmp}, and \textit{Hindsight Experience Replay (HER)}~\cite{her} into a cohesive system.

First, DPTA enhances perception by leveraging Penetrative AI to interpret digitized sensor signals such as force and 
torque, enabling inference of micro-level physical states (e.g., hinge stiffness, surface friction). Second,
 it addresses the limitations of rigid code generation by parameterizing DMPs through Human-Robot Collaboration (HRC), 
 ensuring kinematically sound and human-like motions. Finally, robust learning through HER allows the system to learn from failure, improving policy performance in high-dimensional state spaces.

In summary, DPTA moves beyond the ``Say'' and ``Can'' dichotomy by introducing ``Feel'' and ``Adapt''
 capabilities. By grounding semantic intent in penetrative sensory data and leveraging both stored motion primitives 
 and learned behaviors, DPTA enables embodied agents to plan logically and interact physically with nuance and adaptability.
