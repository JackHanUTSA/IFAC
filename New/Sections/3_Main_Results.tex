\section{Main Results}\label{main}

In this paper,  a new integral controller will be constructed for system (\ref{1}). Firstly, we will modify the technique of adding an integrator to design a novel regulator for a class of chain systems, and then applying the regulator to regulate system (\ref{1}).
\subsection{Global Regulation of Chain Systems} 
\begin{theorem}\label{the1}
	For the following system
	\begin{eqnarray}\label{2}
		\left\{
		\begin{array}{l}
			\dot{z}_{i}=z_{i+1},~i=1,\ldots,n, \\
			\dot{z}_{n+1}=u,
		\end{array}
		\right.
	\end{eqnarray}
there are positive constants $k^{*}$ and $a_i,~i=1,\ldots,n$ such that for any $K(t,~z) \geq k^*$, the linear controller
	\begin{eqnarray}\label{u_z}
		u = -K(t,z)(a_1z_1  + \cdots +  a_{n}z_{n}+ z_{n+1})
	\end{eqnarray}
can globally asymptotically stabilize system (\ref{2}).
\end{theorem}
\textbf{Proof.} Based on the adding a power integrator technique, we propose a constructive method to design a controller (\ref{u_z}) and Lyapunov functions in a recursive manner. \\
\textbf{Step 1:} First, construct
\begin{eqnarray*}\label{z_v1}
	V_1 = \frac{1}{2} z_1^2.
\end{eqnarray*}
The time derivative of $V_1$ along system (\ref{2}) is
\begin{eqnarray}\label{V1_dot}
	\dot V_1|_{(\ref{2})}= z_1z_2 = \epsilon_1z_2^{*} + \epsilon_1(z_2 - z_2^{*})
\end{eqnarray}
with $\epsilon_1:=z_1$. For (\ref{V1_dot}), selecting the virtual controller $z_2^* = -\beta_1\epsilon_1=-(n+1)\epsilon_1$ yields
\begin{eqnarray}\label{V1_dot_c}
	\dot V_1|_{(\ref{2})} = -(n+1)\epsilon_1^2 + \epsilon_1(z_2 - z_2^{*}).
\end{eqnarray}
\textbf{Inductive Step:} 
Suppose at step $k-1$, there is a $C^{1}$ Lyapunov function $V_{k-1}:\mathbb{R}^{k-1}\rightarrow \mathbb{R}$, which is positive definite and proper, and a set of $C^{1}$ virtual controllers $z_{1}^{\ast },\ldots ,z_{k}^{\ast }$, defined by
\begin{eqnarray}\label{7}
\begin{array}{l}
z_{1}^{\ast }=0,
 ~~~~~~~~~~~~~~~\xi _{1}=z_{1}-z_{1}^{\ast }, \\ 
z_{2}^{\ast }=-\beta _{1}\xi _{1}, 
~~~~~~~~~\xi _{2}=z_{2}-z_{2}^{\ast }, \\ 
~~~~\vdots 
~~~~~~~~~~~~~~~~~~~~~~ \vdots \\ 
z_{k}^{\ast }=-\beta _{k-1}\xi _{k-1}, 
~~~ \xi _{k}=z_{k}-z_{k}^{\ast }
\end{array}
\end{eqnarray}
with positive constants $\beta _{1},\ldots ,\beta _{k-1}$ such that 
\begin{eqnarray}\label{8}
\begin{array}{l}
\dot{V}_{k-1}|_{(\ref{2})}
\leq 
-\left( n-k+3\right) \left( \xi _{1}^{2}+\cdots +\xi _{k-1}^{2}\right)\\
~~~~~~~~~~~~~+\xi _{k-1}\left( z_{k}-z_{k}^{\ast}\right).  
\end{array}
\end{eqnarray}
It is clear that (\ref{8}) reduces to the inequality (\ref{V1_dot_c}) when $k=2$ under the definitions of (\ref{7}). Next, we will prove (\ref{8}) also holds at step $k$. To prove this, we choose
\begin{eqnarray}\label{Wk}
	W_k=\frac{1}{2}(z_{k}-z_{k}^{*})^{2}.
\end{eqnarray}
Therefore, for $V_k = V_{k-1} + W_k$, the time derivative of the Lyapunov function $V_{k}$ along system (\ref{2})
is
\begin{eqnarray}\label{11}
\begin{array}{l}
\dot{V}_{k}|_{(\ref{2})} %\\[3mm]
%=\dot{V}_{k-1}+\dot{W}_{k}  \label{11}\\[3mm]
%\leq
%-\left( n-k+3\right) \left( \xi _{1}^{2}+\cdots +\xi_{k-1}^{2}\right)    \\[3mm]
%~~+\xi _{k-1}\left( z_{k}-z_{k}^{\ast }\right) +\frac{\partial W_{k}}{\partial z_{k}}\dot{z}_{k}+\sum_{l=1}^{k-1}\frac{\partial W_{k}}{\partial z_{l}}\dot{z}_{l} \\[3mm]
%\leq
%-\left( n-k+3\right) \left( \xi _{1}^{2}+\cdots +\xi_{k-1}^{2}\right)   \\[3mm]
%~~+\xi _{k-1}\left( z_{k}-z_{k}^{\ast }\right)+\xi _{k}z_{k+1}+\sum_{l=1}^{k-1}\frac{\partial W_{k}}{\partial z_{l}}\dot{
%	z}_{l}  \\[3mm]
\leq
-\left( n-k+3\right) \left( \xi _{1}^{2}+\cdots +\xi_{k-1}^{2}\right)   \\[3mm]
~~~~~~~~~~+\xi _{k-1}\left( z_{k}-z_{k}^{\ast }\right)+\xi _{k}z_{k+1}^{\ast }+\sum_{l=1}^{k-1}\frac{\partial W_{k}}{\partial z_{l}}\dot{z}_{l}\\[3mm]
~~~~~~~~~~+\xi _{k}\left( z_{k+1}-z_{k+1}^{\ast }\right).
\end{array}
\end{eqnarray}

Now we give the estimate of the terms $\xi _{k-1}\left( z_{k}-z_{k}^{\ast }\right) $ and $\sum_{l=1}^{k-1}\frac{\partial W_{k}}{\partial z_{l}}\dot{z%
}_{l}$ in (\ref{11}).

Firstly, from (\ref{7}) and by means of the Young inequality, we have 
\begin{eqnarray} \label{12}
	\xi _{k-1}\left( z_{k}-z_{k}^{\ast }\right) =\xi _{k-1}\xi _{k}\leq \frac{1}{%
		2}\xi _{k-1}^{2}+\frac{1}{2}\xi _{k}^{2}. 
\end{eqnarray}
Next, from (\ref{7}), and
%\begin{eqnarray}\label{13}
%W_{k}& =&\frac{1}{2}\left( z_{k}-z_{k}^{\ast }\right) ^{2}.
%&=&\frac{1}{2}\left( z_{k}+\beta _{k-1}\xi _{k-1}\right) ^{2}  \notag \\
%&\vdots&  \notag \\
%&=&\frac{1}{2}\left( z_{k}+\beta _{k-1}\left( z_{k-1}+\cdots +\beta
%_{2}\left( z_{2}-z_{2}^{\ast }\right) \cdots \right) \right) ^{2}  \notag \\
%&=&\frac{1}{2}\left( z_{k}+\beta _{k-1}\left( z_{k-1}+\cdots +\beta
%_{2}\left( z_{2}+\beta _{1}z_{1}\right) \cdots \right) \right) ^{2}.  \notag
%\end{eqnarray}
 by means of (\ref{Wk}), we have
\begin{eqnarray}\label{14}
\sum_{l=1}^{k-1}\frac{\partial W_{k}}{\partial z_{l}}\dot{z}_{l}
%	=\sum_{l=1}^{k-1}\frac{\partial W_{k}}{\partial z_{l}}z_{l+1}   \\
%&=&\left( z_{k}-z_{k}^{\ast }\right) \sum_{l=1}^{k-1}\beta _{k-1}\cdots
%	\beta _{l}z_{l+1}  \notag \\
%&=&\left( z_{k}-z_{k}^{\ast }\right) \sum_{l=1}^{k-1}\beta _{k-1}\cdots
%	\beta _{l}\left( \left( z_{l+1}-z_{l+1}^{\ast }\right) +z_{l+1}^{\ast
%	}\right)   \notag \\
%&=&\xi _{k}\sum_{l=1}^{k-1}\beta _{k-1}\cdots \beta _{l}\left( \xi_{l+1}-\beta _{l}\xi _{l}\right)   \notag \\
\leq 
\frac{1}{2}\left( \xi _{1}^{2}+\cdots +\xi _{k-1}^{2}\right)+c_{k}\xi _{k}^{2},
\end{eqnarray}%  
where $c_{k}$ is a positive constant. Substituting (\ref{12}) and (\ref{14}) into (\ref{11}) yields
\begin{eqnarray}\label{15} 
\begin{array}{l}
\dot{V}_{k}|_{(\ref{2})}
%         &\leq &-\left( n-k+3\right) \left( \xi _{1}^{2}+\cdots +\xi
%        _{k-1}^{2}\right) +\left( z_{k}-z_{k}^{\ast }\right) z_{k+1}^{\ast } \\
\leq 
 -\left( n-k+2\right) \left( \xi _{1}^{2}+\cdots +\xi_{k-1}^{2}\right)   \\[3mm]
~~~~~~~~~~+\xi _{k}z_{k+1}^{\ast }+\left( c_{k}+\frac{1}{2}\right) \xi _{k}^{2}\\[3mm]
~~~~~~~~~~+\left( z_{k}-z_{k}^{\ast}\right) \left( z_{k+1}-z_{k+1}^{\ast }\right). 
\end{array}
\end{eqnarray}
Now construct the virtual controller%
\begin{eqnarray}\label{16}
\begin{array}{l}
z_{k+1}^{\ast } =-\beta _{k}\xi _{k} %\\[3mm]
=-\left( c_{k}+\frac{1}{2}+n-k+2\right) \xi _{k}
\end{array}
\end{eqnarray}
and substituting (\ref{16})  into (\ref{15}), we have
\begin{eqnarray*}\label{17}
\begin{array}{l}
\dot{V}_{k}|_{(\ref{2})}
\leq
-\left( n-k+2\right) \left( \xi _{1}^{2}+\cdots +\xi_{k-1}^{2}+\xi _{k}^{2}\right) \\[3mm]
~~~~~~~~~~+\left( z_{k}-z_{k}^{\ast }\right) \left(z_{k+1}-z_{k+1}^{\ast }\right).  
\end{array}
\end{eqnarray*}
This completes the inductive proof. \\
\\
\textbf{Last Step:} 
The inductive argument shows that (\ref{8}) holds for $k=n+1$ with a set of virtual controllers (\ref{7}). Based on the inductive argument, we can choose the $(n+1)$th Lyapunov function    
\begin{eqnarray*}\label{18}
\begin{array}{l}
V_{n+1} =V_{n}+W_{n+1}   %\\[3mm]
=V_{n}+\frac{1}{2}\left( z_{n+1}-z_{n+1}^{\ast }\right) ^{2}.
%&=&V_{n}+\frac{1}{2}\xi _{n+1}^{2}.  \notag
\end{array}
\end{eqnarray*}
The time derivative of $V_{n+1}$ along system (\ref{2}) is
\begin{eqnarray}\label{19}
\begin{array}{l}
\dot{V}_{n+1}|_{(\ref{2})} 
%&=&\dot{V}_{n}+\dot{W}_{n+1}   \\
%&\leq &-2\left( \xi _{1}^{2}+\cdots +\xi _{n}^{2}\right) +\xi _{n}\left(
%z_{n+1}-z_{n+1}^{\ast }\right)  \notag \\
%&&+\left( z_{n+1}-z_{n+1}^{\ast }\right) \left( z_{n+1}-z_{n+1}^{\ast}\right) ^{^{\prime }}  \notag \\
\leq 
-2\left( \xi _{1}^{2}+\cdots +\xi _{n}^{2}\right) +\xi_{n+1} u  \\[3mm]
~~~~~~~~~~~~~+\xi _{n}\xi _{n+1}+\left( z_{n+1}-z_{n+1}^{\ast }\right) \left( -\dot{z}_{n+1}^{\ast }\right). 
\end{array}
\end{eqnarray} 
Similar to the estimate of the terms (\ref{12}) and (\ref{14}), we have
\begin{eqnarray}\label{20}
	\xi _{n}\xi _{n+1}\leq \frac{1}{2}\xi _{n}^{2}+\frac{1}{2}\xi _{n+1}^{2}
\end{eqnarray}
and
\begin{eqnarray} \label{21}
\begin{array}{l}
\left( z_{n+1}-z_{n+1}^{\ast }\right) \left( -\dot{z}_{n+1}^{\ast }\right)
\leq 
\frac{1}{2}\left( \xi _{1}^{2}+\cdots +\xi _{n}^{2}\right) \\[3mm]
~~~~~~~~~~~~~~~~~~~~~~~~~~~~~~~~+c_{n+1}\xi_{n+1}^{2}. 
\end{array}
\end{eqnarray}
Substituting (\ref{20}) and (\ref{21}) into (\ref{19}), we have
\begin{eqnarray}\label{22}
\begin{array}{l}
\dot{V}_{n+1}|_{(\ref{2})} 
\leq
-\left( \xi _{1}^{2}+\cdots +\xi _{n}^{2}\right) +\left( c_{n+1}+\frac{1}{2}\right) \xi_{n+1}^{2}\\[3mm]
~~~~~~~~~~~~~~+\xi_{n+1} u. 
\end{array}
\end{eqnarray}

By the adding a power integrator technique, we can simply choose the following controller
\begin{eqnarray}\label{23}
u =-K(t,z)\xi _{n+1},
\end{eqnarray}
where $K(t,z)\geq k^{*}=c_{n+1}+\frac{3}{2}$.
Substituting (\ref{23}) into (\ref{22}), we have
\begin{eqnarray}\label{24}
	\dot{V}_{n+1}|_{(\ref{2})}\leq -\left( \xi _{1}^{2}+\cdots +\xi _{n}^{2}+\xi
	_{n+1}^{2}\right).  
\end{eqnarray}
Thus, we have achieved that system (\ref{2}) is globally asymptotically
stable under controller (\ref{23}), that is 
\begin{eqnarray*} \label{25}
\begin{array}{l}
u =-K(t,z)\xi _{n+1}\\[3mm]
%&=&-K(t,z)\left( z_{n+1}-z_{n+1}^{\ast }\right)  \notag \\
%&=&-K(t,z)\left( z_{n+1}-\left( z_{n+1}^{\ast }\right) \right)  \notag\\
%% &=&-K(t,z)\left( z_{n+1}-\left( -\beta _{n}\xi _{n}\right) \right) 
%% \notag \\
%% &=&-K(t,z)\left( z_{n+1}+\beta _{n}\xi _{n}\right)  \notag \\
%&\vdots&  \notag \\
~~=-K(t,z)( z_{n+1}+\beta_{n}( z_{n}+\cdots +\beta_{2}( z_{2}%\\[3mm]
+\beta_{1}z_{1}) \cdots)),
\end{array}
\end{eqnarray*}    
where $a_{i}=\beta _{i}\cdots\beta _{n},~i=1,\ldots,n$. This completes our proof.\\
\\
Now it is time to present our main results for global regulating system (\ref{1}). First, we utilize Theorem \ref{the1} to solve the regulation problem of chain system (\ref{1}) when $f_i (\cdot)= 0, ~i = 1,\ldots,n$, i.e,
\begin{eqnarray}\label{exp1}
	\left\{
	\begin{array}{l}
		\dot{x}_{i}=x_{i+1},~i=1,\ldots,n-1,   \\
		\dot{x}_{n}=u+d\left( t,x\right). 
	\end{array}
	\right.
\end{eqnarray}
To solve the problem, we assume that the time-varying perturbations with unknown magnitudes $d(t, x)$ satisfies the following assumption.
% assumption 3.1
\begin{assumption}\label{assum1}
Assume there are an unknown constant $\theta $ and a known function $\alpha(t, x) \geq 1$ such that
\begin{eqnarray*}
	d(t, x) = \theta\alpha(t,x).
\end{eqnarray*}
\end{assumption}
\begin{remark}
The uncertain function $d(t,x)$ satisfying Assumption \ref{assum1} encompasses several types of uncertainties in system (\ref{exp1}). First, it includes constant disturbances as its special case when $\alpha(t,x) = 1$. For exogenous time-varying disturbance such as $d(t,x) = c(1 + 0.5\mathrm{sin}(t))$ with unknown magnitude $c$, we can simply choose $\alpha(t,x) = 2(1+ 0.5\mathrm{sin}(t)) \geq 1$ and $\theta = c/2 $. Moreover, $d(t,x)$ can include internal modeling uncertainties such as $d(t,x)=\theta(1+x_2^2)$ with unknown system parameter $\theta$. 	
\end{remark}
%\textbf{Theorem 3.1.2}
\begin{theorem}\label{the2}
Under Assumption \ref{assum1}, there are positive constants $k^{*}$ and $a_i,~i=1,\ldots,n$, such that the following  integral controller
\begin{eqnarray}\label{t1}
	\left\{
	\begin{array}{l}
		u = -k^*\alpha(t,x)(a_1x_0 +a_2x_1+ \cdots \\
		~~~~~+ a_{n}x_{n-1} + x_n ), \\
		\dot x_0 = x_1
	\end{array}
	\right.
\end{eqnarray}
globally regulates system (\ref{exp1}).
\end{theorem}
\textbf{Proof.} Define $z_1 = x_0 - \frac{\theta}{k*a_1}$, $z_i=x_{i-1}, i = 2, \ldots, n+1$. Under the new coordinates, it is clear that the closed-loop system (\ref{exp1}) and (\ref{t1}) can be rewritten as
\begin{eqnarray}\label{exp1_1} 
\dot{z}&=&\left[ 
\begin{array}{c}
		z_{2} \\ 
		\vdots \\ 
		z_{n+1} \\ 
		-k^*\alpha \left( t,x\right) \left( a_{1}z_{1}+\cdots
		+a_{n}z_{n}+z_{n+1}\right)
	\end{array}
	\right]  \nonumber\\
	&=&F\left( t,z\right).        
\end{eqnarray}
By the proof of Theorem \ref{the1}, we can find positive constants $k^*$ and $a_i,~i=1,\ldots,n$ such that for $K(t,z) = k^* \alpha(t,x) \geq k^*$, system (\ref{exp1_1}) is globally regulated. In the case when $d(t,x) = \theta$ is an unknown constant, Theorem \ref{the2} holds for a controller with constant gains.
\begin{corollary}\label{cor1}
The system (\ref{exp1}) with $d(t,x) = \theta$ being an unknown constant $\theta$ can be globally regulated by the integral controller
\begin{eqnarray*}\label{t2}
	\left\{
	\begin{array}{l}
		u = -k^*(a_1x_0 +a_2x_1+ \cdots + a_{n}x_{n-1} + x_n ), \\
		\dot x_0 = x_1
	\end{array}
	\right.
\end{eqnarray*}
for appropriate positive constants $k^*$ and $a_i,~i=1,\ldots,n$.
\end{corollary}
\subsection{Global Regulation of Nonlinear Systems}
This section considers the global regulation problem of system (\ref{1}) when system nonlinearities satisfy the following assumption.
%Assumptions 3.2
\begin{assumption}\label{assum2} 
For $i=1,\ldots ,n$, there is a known constant $c$ such that
\begin{eqnarray*}\label{31}
	\left\vert f_{i}\left( x_{1},\ldots ,x_{i}\right) \right\vert \leq c\left(\left\vert x_{1}\right\vert +\cdots +\left\vert x_{i}\right\vert \right).  
\end{eqnarray*}
\end{assumption}
\begin{theorem}\label{the3}
Under Assumptions \ref{assum1} and \ref{assum2}, there are positive constants $k^*$ and $a_1, \ldots, a_n$ such that for a large enough constant $L\geq 1$, the following integral controller
\begin{eqnarray} \label{32}
\left\{
\begin{array}{l}
u =-L^{n}k^*\alpha \left( t,x\right) \big(a_{1}x_{0}+a_{2}x_{1}+a_{3}\frac{x_{2}}{L} \\[3mm]
~~~~~+\cdots+a_{n}\frac{x_{n-1}}{L^{n-2}}+\frac{x_{n}}{L^{n-1}}\big),   \\[3mm]
\dot{x}_{0} = Lx_{1}  
\end{array}
\right.
\end{eqnarray}
with $\alpha(t,x)$ defined in Assumption \ref{assum1} solves the global regulation problem of system (\ref{1}).
\end{theorem}
\textbf{Proof.} Define $z_1 = x_0 - \frac{\theta}{k*a_1L^n}$, $z_{i}=\frac{x_{i-1}}{L^{i-2}},~ i = 2, \ldots, n+1$. By choosing the same constants $k^*$ and $a_i,~i =1, \ldots, n$ as in Theorem \ref{the2}, the closed-loop system (\ref{1}) and (\ref{32}) under the new coordinates can be rewritten as
\begin{eqnarray*}\label{33}
\left\{
\begin{array}{l}
\dot{z}_{1}=Lz_{2},\\ 
\dot{z}_{2}=Lz_{3}+f_{1}(x_{1}),\\ 
\dot{z}_{3}=Lz_{4}+\frac{f_{2}(x_{1},x_{2})}{L},\\ 
~~~~~\vdots  \\ 
\dot{z}_{n}=Lz_{n+1}+\frac{f_{n-1}(x_{1},\ldots,x_{n-1})}{L^{n-2}},\\ 
\dot{z}_{n+1}=-Lk^*\alpha \left( t,x\right) \left( a_{1}z_{1}+\cdots
+a_{n}z_{n}+z_{n+1}\right)\\
~~~~~~~~~+\frac{f_{n}(x_{1},\ldots,x_{n})}{L^{n-1}},\\ 
\end{array}
\right.
\end{eqnarray*}
which can be further rewritten as the following matrix form
\begin{eqnarray}\label{34}
	\dot{z}=LF\left( t,z\right) +\varPhi,
\end{eqnarray}
where $F(t,z)$ is the same as the one in (\ref{exp1_1}) and $\varPhi=\big(0,f_{1}\left( x_{1}\right),\frac{f_{2}\left( x_{1},x_{2}\right) }{L},\ldots,\frac{f_{n}\left( x_{1},\ldots ,x_{n}\right) }{L^{n-1}}\big)^{T}$. 




By using the same Lyapunov function  $V_{n+1}$ constructed in Theorem \ref{the1}, we can see that the time derivative of $V_{n+1}$ along system (\ref{34}) is
\begin{eqnarray}\label{35}
\begin{array}{l}
\dot{V}_{n+1}|_{(\ref{34})} \\[3mm]
=\frac{\partial V_{n+1}}{\partial z}\left( LF\left( t,z\right) +\varPhi \right)\\[3mm]
%=L\frac{\partial V_{n+1}}{\partial z}F\left( t,z\right) +\sum_{i=2}^{n+1}
%	\frac{\partial V_{n+1}}{\partial z_{i}}\frac{f_{i-1}\left( x_{1},\ldots
%		,x_{i-1}\right) }{L^{i-2}}  \\[3mm]
\leq 
-L\sum_{l=1}^{n+1}\xi _{l}^{2}+\sum_{i=2}^{n+1}\sum_{l=i}^{n+1}\frac{\partial V_{l}}{\partial z_{i}}\frac{f_{i-1}\left( x_{1},\ldots,x_{i-1}\right)}{L^{i-2}}. 
\end{array}
\end{eqnarray}
By Assumption \ref{assum2} and the fact $L\geq 1$, we have  
\begin{eqnarray}\label{36}
\begin{array}{l}
\left\vert \frac{f_{i-1}(\cdot) }{L^{i-2}}\right\vert  
\leq 
\frac{c\left( \left\vert x_{1}\right\vert+\cdots +\left\vert x_{i-1}\right\vert \right) }{L^{i-2}}   %\\[3mm]
%~~~~~~~~~~
%\leq 
%c\left( \left\vert z_{2}\right\vert +\cdots +\left\vert z_{i}\right\vert \right) \\[3mm]
%~~~~~~~~~~
\leq 
\tilde{c}\left( \left\vert \xi _{1}\right\vert +\cdots +\left\vert \xi_{i}\right\vert \right) 
\end{array}
\end{eqnarray}
for a positive constant $\tilde{c}$.
In addition, from the definition of system (\ref{13}), we have
\begin{eqnarray} \label{37}
\begin{array}{l}
\sum_{l=i}^{n+1}\frac{\partial V_{l}}{\partial z_{i}}\leq \bar{c}%
\sum_{l=i}^{n+1}\left\vert \xi _{i}\right\vert
\end{array}  
\end{eqnarray}
for a positive constant $\bar{c}$.
Then, by substituting (\ref{36}) and (\ref{37}) into (\ref{35}), we have%
\begin{eqnarray*} \label{38}
\begin{array}{l}
\dot{V}_{n+1}|_{(\ref{34})} % \\[3mm]
%\leq 
%-L\sum_{l=1}^{n+1}\xi_{l}^{2} \\[3mm]
%~~+\sum_{i=2}^{n+1}\bar{c}\sum_{l=i}^{n+1}\left\vert \xi_{i}\right\vert \tilde{c}\left( \left\vert \xi _{1}\right\vert +\cdots+\left\vert \xi _{i}\right\vert \right)  \\[3mm]
\leq 
-L\sum_{l=1}^{n+1}\xi _{l}^{2}+\hat{c}\sum_{l=1}^{n+1}\xi _{l}^{2}
\end{array}
\end{eqnarray*}%
for a positive constant $\hat{c}$.

Selecting a large enough $L \geq \hat{c}+1$, we can get a relation same as (\ref{24}). As a result, the integral controller (\ref{32}) can regulate system (\ref{1}) under Assumptions \ref{assum1} and \ref{assum2}.
%\textbf{Remark 3.2.1}
%\begin{remark}\label{chrm1}
%In this section, a static high-gain parameter $L$ is introduced to achieve the global regulation problem for system (\ref{1}) under Assumptions \ref{assum1} and \ref{assum2}. The aim of introducing $L$ is to tackle the nonlinear terms $ f_i(\cdot),~i=1,\ldots,n$ by expanding the gain parameters $k^{*}$ and $a_i,~i=1,\ldots,n$ given in Theorem \ref{the2}, $L$ times, which makes the proof of Theorem \ref{the3} more concise. On the other hand, we can also achieve the global regulation of system (\ref{1}) under Assumptions \ref{assum1} and \ref{assum2} by following the proof of Theorems \ref{the1} and \ref{the2}. Specifically speaking, we can estimate the nonlinear term in each step of the proof process of Theorem  \ref{the1}, and then re-select gain parameters $k^{*}$ and $a_i,~i=1,\ldots,n$, rather than introducing $L$, to achieve our control objective.
%\end{remark}
%In order to elaborate Remark \ref{chrm1}, the following example is introduced.
%\textbf{Example 3.1}
%\textbf{Example 3.2.1}





\subsection{Extension}
We reconsider system (\ref{1}) where system vanishing uncertainties $f_{i}(x_{1},\ldots,x_{i}),~i=1,2,\ldots,n$ satisfy Assumption  \ref{assum2} and the non-vanishing uncertainty $d(t, x) $ satisfies the following assumption.

%\textbf{Assumption 3} 
\begin{assumption}\label{assum3}
%
Assume the non-vanishing uncertainty $d(t, x)$ satisfies
\begin{eqnarray*}
	d(t, x) = \theta\alpha(t,x),
\end{eqnarray*}
where $\theta $ is an unknown constant and $\alpha(t, x)$ is a continuous function. Moreover,  $\alpha(t, x)$ also satisfies
$\alpha(t, x) \geq \bar{\alpha}(t) \geq 0$ where the continuous function $\bar{\alpha}(t)$ is a periodic function with $T>0$ being the period.
%where $\theta $ is an \textit{unknown} constant and $\alpha(t, x)$ is a continuous function. Moreover,  $\alpha(t, x)$ also satisfies
%$\alpha(t, x) \geq \bar{\alpha}(t)$ where the continuous function $\bar{\alpha}(t)$ is a periodic function and satisfies $\int_{0}^{T}\bar{\alpha}(s)> \sigma $ with $T>0$ being the period of $\bar{\alpha}(t)$ and $\sigma$ being a positive constant.
\end{assumption}




\begin{remark}
Assumption \ref{assum3} is significantly different from Assumption \ref{assum1}, and brings great difficulties to the stability analysis of system (\ref{1}). Specifically speaking, due to the condition $\alpha(t, x) \geq 1$ in Assumption \ref{assum1}, there must be a constant $k^*$ in Theorem \ref{the2} such that $k^*\alpha(t, x) > c_{n+1}+\frac{1}{2}$ for all $(t, x) \in \mathbb{R}_{\geq 0} \times \mathbb{R}^{n}$. However, since $\alpha(t, x) \geq \bar{\alpha}(t) \geq 0$ holds in Assumption \ref{assum3} and for all $(t, x) \in \mathbb{R}_{\geq 0} \times \mathbb{R}^{n}$, we cannot choose the constant $k^*$, such that $k^*\alpha(t, x) > c_{n+1}+\frac{1}{2}$, which brings great challenges to the stability analysis of the closed-loop system.
\end{remark}

Based on Assumptions \ref{assum2} and \ref{assum3}, we can have the following result.
%\textbf{Theorem 5.1} 
\begin{theorem}\label{the4}
%
Under Assumptions \ref{assum2} and \ref{assum3}, there are positive constants $k^*$ and $a_1, \ldots, a_n$ such that for a large enough constant $L\geq 1$, the following integral controller
\begin{eqnarray} \label{e32}
\left\{
\begin{array}{l}
u =-L^{n}k^* \alpha(t,x) \big(a_{1}x_{0}+a_{2}x_{1}+\cdots+a_{n}\frac{x_{n-1}}{L^{n-2}}\\[3mm]
~~~~~+\frac{x_{n}}{L^{n-1}}\big),  \\[3mm]
\dot{x}_{0} = Lx_{1}  
	\end{array}
	\right.
\end{eqnarray}
with $\alpha(t,x)$ defined in Assumption \ref{assum3} solves the global regulation problem of system (\ref{1}).
\end{theorem}
\textbf{Proof.} Following the proof of Theorems \ref{the1}, \ref{the2} and \ref{the3}, we can easily obtain 
\begin{eqnarray}\label{e1} 
	\dot{V}_{n+1}|_{(\ref{34})} 
	\leq -L\sum_{l=1}^{n}\xi _{l}^{2}+C\xi _{n+l}^{2} +\xi _{n+1}u,
\end{eqnarray}
where $C$ is a positive constant independent of $L$. From (\ref{e32}), we construct the controller 
\begin{eqnarray}\label{e2} 
	u=-L^{2}k^{*}\alpha(t, x)\xi _{n+1}
\end{eqnarray}
and substituting (\ref{e2}) into (\ref{e1}), we have
\begin{eqnarray}\label{e3} 
	\dot{V}_{n+1}|_{(\ref{34})} 
	\leq -L\sum_{l=1}^{n}\xi _{l}^{2}+C\xi _{n+1}^{2} -L^{2}k^{*}\alpha(t, x)\xi _{n+1}^{2},
\end{eqnarray}
and by means of Assumption \ref{assum3} and $L\geq 1$, we further obtain
\begin{eqnarray}\label{e4} 
	\dot{V}_{n+1}|_{(\ref{34})} 
	\leq 
	-L\sum_{l=1}^{n}\xi _{l}^{2}-(Lk^{*}\bar{\alpha}(t)-C)\xi _{n+l}^{2}.
\end{eqnarray}
Then, from Assumption \ref{assum3}, we have $\bar{\alpha}(t) \geq 0$. If $\bar{\alpha}(t) > 0$ for $t\in[0,T]$, and following the continuous of  $\bar{\alpha}(t)$ and the proof of Theorem \ref{the3}, we can easily prove Theorem \ref{the4}.

If there exists a time $t^{*}\in [0,T]$ such that $\bar{\alpha}(t^{*}) = 0$ and without loss of generality, we assume that only $t^{*}\in [0,T]$ exists making $\bar{\alpha}(t^{*}) = 0$. Firstly, define $T_{1}=\{\bar{\alpha}(t) \leq \varepsilon | ~t\in [0,T]\}$, $T_{2}=\{\varepsilon< \bar{\alpha}(t) \leq L | ~t\in [0,T]\}$ and $T_{3}=\{\bar{\alpha}(t) \geq L | ~t\in [0,T]\}$, where $\varepsilon$ is a small positive constant and $T_{3}$ can be an empty set. We can get that $t^{*} \in T_{1}$ and form (\ref{e4}), we have
\begin{eqnarray}\label{e5} 
	\dot{V}_{n+1}|_{(\ref{34})} 
	\leq 
	\left\{
	\begin{array}{l}
		C\sum_{l=1}^{n+1}\xi _{l}^{2},~~~~~~~~~~~~~~~~~~~t\in T_{1},\\
		-(Lk^{*}\bar{\alpha}(t)-C)\sum_{l=1}^{n+1}\xi _{l}^{2}, ~t\in T_{2},\\
		-L\sum_{l=1}^{n+1}\xi _{l}^{2}, ~~~~~~~~~~~~~~~~~t\in T_{3}.\\
	\end{array}
	\right.
\end{eqnarray}
By means of $\varepsilon< \bar{\alpha}(t) \leq L$ for $t \in T_{2}$ and from (\ref{7}), (\ref{e5}) can be further rewritten as
\begin{eqnarray}\label{e6} 
	\dot{V}_{n+1}|_{(\ref{34})} 
	\leq 
	\left\{
	\begin{array}{l}
		2CV_{n+1},~~~~~~~~~~~~~~~~~t\in T_{1},\\
		-2(Lk^{*}\varepsilon-C)V_{n+1}, ~~~t\in T_{2},\\
		-2LV_{n+1}, ~~~~~~~~~~~~~~~t\in T_{3}.\\
	\end{array}
	\right.
\end{eqnarray}

Integrating both sides of inequality (\ref{e6}) from $0$ to $T$, we have
\begin{eqnarray}\label{e7}
	V(T) 
	\leq 
	V(0) e^{2\left(\int_{T_{1}}Cds-\int_{T_{2}}(Lk^{*}\varepsilon-C)ds-\int_{T_{3}}Lds\right)}.
\end{eqnarray}
From (\ref{e7}), we  can get a large $L\geq 1$, such that $Lk^{*}\varepsilon-C>0$ and $\int_{T_{1}}Cds<\int_{T_{2}}(Lk^{*}\varepsilon-C)ds$ hold, which indicates $V(T) \leq V(0) $. On the other hand, since $\bar{\alpha}(t)$ is a periodic function with $T>0$ being the period, we can eventually get $\lim_{t \rightarrow +\infty}V(t)=0$. By Lemma \ref{lemma 2.2}, we can achieve the proof of Theorem \ref{the4}.










%\begin{eqnarray}\label{s6}
%\dot{x}=
%\left[
%\begin{array}{l}
%	x_{2}^{p}\\
%	x_{3}\\
%	-a_{3}\left(a_{1}x_{1}+a_{2}x_{2}^{\frac{p+1}{2}}+x_{3}\right)^{\frac{2p}{p+1}}
%\end{array}
%\right]
%=H(t,x).
%\end{eqnarray}

