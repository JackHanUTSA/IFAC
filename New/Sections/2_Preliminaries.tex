%%%%%%%%%%%%%%%%%%%%%%%%%%%%%%%%%%%%%%%%%%%%%%%%%%%%%%%%%%%%%%%%%%%%%%%%%%%%%%%%%%
\section{Preliminaries}\label{preliminaries}

In order to get our main results, the following lemma is introduced firstly.
%\begin{lemma}\label{lemma 2.1}
%Let $c, ~d$ be positive constants and for any positive constant $\epsilon \geq 0$, the following inequality holds
%\begin{eqnarray}
%	|x|^c|y|^d \leq \frac{c}{c+d}\epsilon|x|^{c+d} + \frac{d}{c+d}\epsilon^{-\frac{c}{d}}|y|^{c+d}.
%\end{eqnarray}
%\end{lemma}

\begin{lemma}\label{lemma 2.2}
Consider the time-varying nonlinear system 
\begin{eqnarray}\label{s02}
\dot{x}=f(t,x),
\end{eqnarray}
where $f(t,x):\mathbb{R}_{\geq 0} \times \mathcal{D}\rightarrow \mathbb{R}^{n}$ is continuous with respect to $x$ in an open neighborhood $ \mathcal{D}\subseteq \mathbb{R}^{n}$ and piecewise continuous with respect to $t$, $f(t,0)=0,~t\in \mathbb{R}_{\geq 0}$ and the initial state is $x(0)=x_{0} \in \mathcal{D}$. If there is a continuous, positive definite function $V(x):\mathcal{D} \rightarrow \mathbb{R}_{\geq 0}$ for system (\ref{s02}) such that
	\begin{eqnarray}\label{s03}
	\dot{V}(x) \leq -\mu(t)V,
	\end{eqnarray}
where $\mu(t)$ is piecewise continuous (containing removable discontinuities and jump discontinuities)  satisfying 
	\begin{eqnarray}\label{s04}
	\lim_{t \rightarrow +\infty}\int_{0}^{t} \mu(s)ds =+\infty.
	\end{eqnarray}
Then, the trajectories of system (\ref{s02}) is locally convergent to zero.
Furthermore, if $\mathcal{D}= \mathbb{R}^{n}$, the trajectories of system (\ref{s02}) is also globally convergent to zero.
\end{lemma}
\textbf{Proof.} Firstly, integrating both sides of inequality (\ref{s03}) from $0$ to $t$, we can easily have
\begin{eqnarray*}\label{s05}
	V(t) \leq V(x_{0})e^{-\int_{0}^{t} \mu(s)ds}.
\end{eqnarray*}
From the condition (\ref{s04}), we can get that system (\ref{s02}) is locally convergent to zero. \\
%%%%%%%%%%%%%%%%%%%%%%%%%%%%
\begin{remark}\label{rem1}
Lemma \ref{lemma 2.2} extends the existing Lyapunov conditions of asymptotically stability since it will degenerate to the results when $\mu(t),~t \in \mathbb{R}_{\geq 0}$ being a negative constant, i.e., there exists a positive constant $c$ such that $\mu(t) \leq c,~t \in \mathbb{R}_{\geq 0}$. However, the criterion can allow  $\mu(t),~t \in \mathbb{R}_{\geq 0}$ is equal to zero and even positive at some time points, which makes the positive constant $c$ not exist.
We introduce the following example to further elaborate Lemma \ref{lemma 2.2}  and Remark \ref{rem1}. \\	
\end{remark}

\begin{example}\label{exam1}
Consider a scalar time-varying nonlinear system
\begin{eqnarray}\label{s06}
	\dot{x}(t)=-\mu(t)x(t),~t \in \mathbb{R}_{\geq 0},
\end{eqnarray}
where $x(0)=x_{0}>0$ and the following two cases are analyzed based on system (\ref{s06}). \\

\textbf{Case 1.} When $\mu(t)=(1+\mathrm{cos}t)$ is a continuous function with respect to $t$. Firstly, from the definition of $\mu(t)$, we get that $\mu(t)$ is equal to zero in $T_{1}=\{t| t=\pi + 2k\pi,~ t\in \mathbb{R}_{\geq 0},~ k \in \mathbb{Z}_{\geq 0}\}$ and positive in $t\in \mathbb{R}_{\geq 0}/T_{1}$. Then, we can easily obtain that  $\lim_{t \rightarrow +\infty}\int_{0}^{t} \mu(s)ds =+\infty$. Thus, we can get that the trajectories of system (\ref{s06}) is globally convergent to zero.

\textbf{Case 2.} When $\mu(t)=\left(\frac{1}{2}+\mathrm{sin}t\right)$ is a continuous function with respect to $t$.
Firstly, we get that $\mu(t)$ is a negative function in $T_{2}=\big\{t|t\in \big(\frac{7}{6}\pi + 2k\pi, \frac{11}{6}\pi + 2k\pi \big), ~ t\in \mathbb{R}_{\geq 0}, ~k \in \mathbb{Z}_{\geq 0}  \big\}$.
Then, integrating both sides of system (\ref{s06}) from $0$ to $t$, we have $\int_{0}^{t} \mu(s)ds=\left(\frac{1}{2}t-\mathrm{cos}t +1\right)$ is equal to zero at $t=0$, positive when $t>0$ and converges to $+\infty$ as $t \rightarrow +\infty$, which indicates that the trajectories of system (\ref{s06}) is globally convergent to zero.	
\end{example}




%\begin{lemma}\label{lem2}
%	For $x\in \mathbb{R}$, $y\in \mathbb{R}$, $p\geq 1$ is a constant, the following inequalities hold 
%	\begin{eqnarray*}
%		&&|x+y|^{p}\leq 2^{p-1}|x^{p}+y^{p}|,   \\[2mm]
%		&&(|x|+|y|)^{\frac{1}{p}}\leq |x|^{\frac{1}{p}}+|y|^{\frac{1}{p}}.
%	\end{eqnarray*}
%	If $p\geq 1$ is an odd integer, then 
%	\begin{eqnarray*}
%		\left|x^{\frac{1}{p}}-y^{\frac{1}{p}}\right| \leq 2^{1-\frac{1}{p}}|x-y|^{\frac{1}{p}}.
%	\end{eqnarray*}
%\end{lemma}
%
%\begin{lemma}\label{lem3}
%	Let $c,d$ be positive constant. Given any positive number $\gamma >0$, the following inequality holds 
%	\begin{eqnarray}
%		|x|^{c}|y|^{d} \leq \frac{c}{c+d}\gamma|x|^{c+d}+\frac{d}{c+d}\gamma^{-\frac{c}{d}}|y|^{c+d}.  \nonumber
%	\end{eqnarray}
%\end{lemma}

%%%%%%%%%%%%%%%%%%%%%%%%%%%%%%%%%%%%%%%%%%%%%%%%%%%%%%%%%%%%%%%%%%%%%%

