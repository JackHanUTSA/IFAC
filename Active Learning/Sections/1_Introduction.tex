\section{Introduction}\label{introduction}

This is the most critical area for "Focus Mode." In creative writing, an LLM hallucination is a funny plot twist. In robotics, a hallucination is a crash.

When controlling hardware (like your e-puck2), the "crossing paths" problem you mentioned is dangerous. If the LLM starts "thinking" about the philosophy of movement instead of outputting a motor velocity, latency spikes and the robot hits a wall.

Here is how we apply the Focus Mode architecture to robotic control to ensure safety and precision.


1. The Architecture: Split the Brain
We cannot have one LLM doing everything. We must split the control loop into two distinct agents with "Focused" system prompts.

Agent A: The Strategist (High Level)

Focus: Logic, planning, sequence, environment analysis.

Input: "Find the red object."

Output: A list of sub-goals (e.g., "Rotate 90 degrees," "Move forward 10cm").

Agent B: The Driver (Low Level - "Focus Mode")

Focus: Strictly translating sub-goals into hardware commands (JSON).

Input: "Rotate 90 degrees."

Output: {"left_motor": 500, "right_motor": -500, "duration_ms": 1000}.